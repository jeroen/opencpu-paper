% !TEX root = main.tex
\section{The OpenCPU system}

This section introduces the \OpenCPU system. At this point the general concerns become more concrete as we illustrate how the pieces come together in the context of \R and \HTTP. The intention of this paper is not to describe every feature of the \API in great detail. The online documentation and reference implementations are the best source specific information required for building clients and applications. This paper serves as a conceptual introduction, and focuses on the main parts of the interface that exemplify the separation of concerns central to this work. 

%The parts that were left out of the interface specification are just as interesting as the \API itself. For example the \RPC mechanism implies that concerns of managing processes, hardware and security are at the discretion of the server. Moreover, because clients can specify the data interchange format for retrieving an arbitrary resource, the server needs to implement mappings between the various resource types and data interchange formats. And the fact that computational exceptions are returned using \HTTP status codes implies that error handling is the responsibility of the client. These are some examples of how the \API implicitly gives rise to the boundaries for separation concerns and suggest a minimal structure for embedded scientific computing. 

\subsection{Resource types}

As was described earlier, individual requests within the \OpenCPU \API are stateless and there is no notion of a \emph{process}. State of the system changes through creation and manipulation of resources. We start by describing the various resource types which are the conceptual building blocks of the \API. Each resource type has unique properties and supports different operations.

\subsubsection{Objects}

Objects are the main entities of the system and carry the same meaning as within a functional language. They include data structures, functions, or other types supported by the back-end language, in our case \R. Each object has an individual endpoint within the \API and unique name or key within its namespace. The client needs no knowledge of the implementation of these objects. Analogous to a \UI, the primary purpose of the \API is managing objects (creating, retrieving, publishing) and performing procedure calls. Existing objects may also be used as arguments in a remote function call. Objects created from executing a script or returned by a function call are automatically stored and gain the same status as other existing objects. The \API does not distinguish between static objects that appear in e.g. packages, or dynamic objects created by users, nor does it distinguish between objects in memory or on disk. The \API merely provides a system for referencing objects in a way that allows clients to control and reuse them. The implementation of persistence, caching and expiration of objects is at the discretion of the server. 

\subsubsection{Namespaces}

A namespace is a collection of uniquely named objects with a given path in the \API. In \R, static namespaces are implemented using \emph{packages} and dynamic namespaces exist in \emph{environments} such as the user workspace. However \OpenCPU abstracts the concept of a namespace as a set of uniquely named objects and does not distinguish between static or dynamic, persistent or temporary namespaces. Clients can request a list of the contents of any namespace, however the server might refuse such a request for private namespaces or hidden objects. 

\subsubsection{Formats}

\OpenCPU explicitly differentiates a resource from a \emph{representation} of that resource in a particular \emph{format}. The \API lets the client rather than the server decide on the format used to serve content. This is a difference with common scientific practices of exchanging data, documents and figures in fixed format files. Resources in \OpenCPU can be retrieved using various output formats and formatting parameters. For example a basic dataset can be retrieved in \texttt{csv}, \texttt{json}, \texttt{Protocol Buffers} or \texttt{tab delimited} format. Similarly, a graphic can be retrieved in \texttt{svg}, \texttt{png} or \texttt{pdf} and manual pages can be retrieved in \texttt{text}, \texttt{html} or \texttt{pdf} format. In addition to the format, the client can specify formatting parameters in the request. The system supports many additional formats, but not every format is appropriate for every resource type. When a client requests a resource in a format using an invalid format, the server responds with an error.   


\subsubsection{Data}

The \API defines a seperate entity for \emph{data} objects. Even though data can technically be treated as general objects, in practice they often serve a different purpose. For example data are usually not language specific and can not be called or executed. Therefore it is often suitable to conceptually distinguish this subset of objects. For example \R uses lazy loading of data objects to save memory when for packages containing large datasets.

\subsubsection{Graphics}

Any function call can produce zero or more graphics. After completing a remote function call, the server reports how many graphics were created and provides the key for referencing these graphics. Clients can retrieve each individual graphic in subsequent requests using one of various output formats such as \texttt{png}, \texttt{pdf}, and \texttt{svg}. Where appropriate the client can specify additional formatting paramters during the retrieval of the graphic such as width, height or font size.

\subsubsection{Files}

Files can be uploaded and downloaded using standard \HTTP mechanics. The client can post a file as an argument in a remote function call, or download files that were saved to the working directory by the function call. Support for files also allows for hosting web pages (e.g. \texttt{html}, \texttt{css}, \texttt{js}) that interact with local \API endpoints to serve a web application. Furthermore files can be used as arguments to remote function calls, and files that are recognized as \emph{scripts} can be executed using \RPC.

\subsubsection{Manuals}

In most scientific computing languages, each function or dataset that is available to the user is accompanied by an identically named manual page. This manual page includes information such as description and usage of functions and their arguments, or explanation and comments about the columns of a particular dataset. Manual pages can be retrieved in various formats including plain text, \texttt{html} and \texttt{pdf}.

\subsubsection{Containers}

We refer to a path on the server containing one or more collections of resources as a \emph{container}. The current implementation supports two types of containers. A \emph{package} is a static container which may include a namespace with \R objects, manual pages, data and files.  A \emph{session} is another type of container that holds outputs created from executing a script or function call, including a namespace with \R objects, graphics and files. However the distinction between packages and sessions is actually an implementation detail. The \API does not differentiate between the various container types: interacting with an object or file works the same, regardless of whether it is part of a package or session. Future implementations or other servers might use other container types for grouping a collection of resources.

\subsubsection{Libraries}

We will refer to a collection of containers as a \emph{library}. In \R terminology a library is a directory on disk with installed packages. However within the context of the \API, the concept is not limited to packages but refers more generally to any set of containers. For example the \texttt{/ocpu/tmp/} library is the collection of temporary sessions. Also the \API notion of a library does not require containers to be preinstalled. For example, a remote collection of packages, which in \R terminology is called a \emph{respository}, can also be implemented as a library. The current implementation of \OpenCPU exposes the \texttt{/ocpu/cran/} library which refers to the current packages on the \texttt{CRAN} repository. The \API does not differentiate between interfacing a library of sessions, locally installed packages or remote packages. Interacting with an object from a \texttt{CRAN} package works the same as interacting with an object from a local package or session. The \API leaves it up to the server which types of libraries it wishes to expose and how to implement this. The current version of \OpenCPU uses a combination of cronjobs and on-the-fly package installations to synchronize packages on the server with the \texttt{CRAN} repositories.

\subsection{On HTTP}

One of the major strengths of \OpenCPU is that it builds on the hypertext transfer protocol. \HTTP is the most used application protocol on the internet, and the foundation of data communication in browsers and the world wide web. The \HTTP specification is very mature and widely implemented. It provides all functionality required to build modern applications and has recently gained popularity for web \API's as well. The benefit of using a standardized application protocol such as \HTTP is that a lot of funtionality gets built-in by design. \HTTP has excellent mechanisms for authentication, encryption, caching, distribution, concurrency, error handling, etc. This allows us to defer most appliation logic of our system to the protocol and limit the \API specification to logic of scientific computing. 

The \OpenCPU \API defines a mapping between \HTTP requests and high-level operations such as calling functions, running scripts, access to data, manual pages and management of files and objects. The \API specifically does not prescribe any language implementation details. Syntax and low-level concerns such as process management or code evaluation are abstracted and at the discretion of the server implementation. The \API also does not describe any logic which can be taken care of on the protocol or application layer. For example, to add support for authentication, any of the standard mechanisms can be used such as \texttt{basic auth} \citep{franks1999rfc} or \texttt{OAuth 2.0} \citep{hardt2012oauth}. The implementation of such authentication methods might vary from a simple server configuration to defining additional endpoints. However authentication will not affect the meaning of the \API itself and can therefore considered independent of this research. The same holds for other native features of the \HTTP protocol which can be used in conjunction with the \OpenCPU \API (or any other \HTTP \API for that matter). 

What remains after cutting away implementation and application logic is a simple and interoperable interface that is easy to understand and can be implemented with standard \HTTP software libraries. This is an enormous advantage over many other bridges to \R and critical to make the system scalable and extensible. 

\subsubsection{HTTP Methods}

The current \API uses two \HTTP methods: \GET and \POST. As per \HTTP standards, \GET is a \emph{safe} method which means it is intended only for information retrieval and should not change the state of the server. The \GET method is used to retrieve objects, manuals, graphics or files. The parameters of the request are mapped to the formatting function. A \GET requests targeting a container, namespace or directory is used to list the contents. The \POST method on the other is used for \RPC which does change the server state. A \POST request targeting a function results in a remote function call where the \HTTP parameters are mapped to function arguments. A \POST request targeting a script results in an execution of the script where \HTTP parameters are mapped to the script interpreter. Table \ref{table:methods} gives an overview of these uses, which uses the \texttt{MASS} package \citep{MASS} as an example.

\begin{table}[H]
\def\arraystretch{1.5}%
\begin{tabular}{@{}lllll@{}}
\toprule
\emph{Method} & \emph{Target} & \emph{Action}  & \emph{Parameters}     & \emph{Example}                                      \\ \midrule
\texttt{GET}    & object  & retrieve      &  formatting     & \texttt{GET /ocpu/library/datasets/R/mtcars/json}            \\
                & manual  & read          &  formatting     & \texttt{GET /ocpu/library/MASS/man/rlm/html}            \\  
                & graphic & render        &  formatting    & \texttt{GET /ocpu/tmp/\{key\}/graphics/1/png}            \\   
                & file    & download      & -                     & \texttt{GET /ocpu/library/MASS/NEWS}                         \\
                & path    & list contents & -                     & \texttt{GET /ocpu/library/MASS/scripts/}                     \\ \midrule
\texttt{POST}   & object  & call function & function arguments    & \texttt{POST /ocpu/library/stats/R/rnorm}                    \\
                & file    & run script    & control interpreter   & \texttt{POST /ocpu/library/MASS/scripts/ch01.R}              \\ \bottomrule
\end{tabular}
\caption{Currently implemented \HTTP methods}
\label{table:methods}
\end{table}

\subsubsection{HTTP Status codes}

Each \HTTP response includes a status code. Table \ref{table:statuscodes} lists some common \HTTP status codes used by \OpenCPU that the client should be able to interpret. The meaning of these status codes is conform \HTTP standards. The web server might use additional status codes that are not specific to \OpenCPU for more general purposes.

\begin{table}[H]
\centering
\def\arraystretch{1.5}%
%\noindent\begin{tabular}{\columnwidth}{ *{3}{X} }
\begin{tabular}{@{}lll@{}}
\toprule
\emph{Status Code}              & \emph{Happens when}                             & \emph{Response content}                     \\ \midrule
\texttt{200 OK}          & On successful \texttt{GET} request                     & Requested data                    \\
\texttt{201 Created}     & On successful \texttt{POST} request                    & Output key and location                     \\
\texttt{302 Found}       & Redirect                                               & Redirect location                   \\
\texttt{400 Bad Request} & On computational error in \R                                     & Error message in \texttt{text/plain} \\
\texttt{502 Bad Gateway} & Back-end server offline                            & -- (See error logs) \\
\texttt{503 Bad Request} & Back-end server failure                                & -- (See error logs) \\ \bottomrule                          
\end{tabular}
\caption{Commonly used \HTTP status codes}
\label{table:statuscodes}
\end{table}

\subsubsection{HTTP Content-types}

Clients can retrieve objects on the server in various \emph{formats} by postfixing the object \URL with the format identifier in the \GET request. Which formats are supported and how object types are mapped to a particular format is at the discretion of the server implementation. Not every format has to support any object type. For example, \texttt{csv} can only be used to retrieve tabular data structures, and \texttt{png} is only appropriate for graphics. Table \ref{table:formats} lists the formats \OpenCPU supports, their standard internet media type identifier, and the \R function that our implementation uses to export an object into a particular format. Arguments of the \GET requests are mapped to the respective export function. For example the \texttt{png} format has parameters such as \texttt{width} and \texttt{height}, whereas the \texttt{tab delimited} format has parameters \texttt{sep}, \texttt{eol}, \texttt{dec} which specify the delimiting, end-of-line and decimal character respectively.

\begin{table}[H]
\def\arraystretch{1.5}%
\begin{tabular}{@{}lllll@{}}
\toprule
 \emph{Format} & \emph{Content-type}             & \emph{Export function}      & \emph{Example}    \\ \midrule
 \texttt{print}  & \texttt{text/plain}               & \texttt{base::print}    & \texttt{/ocpu/cran/MASS/R/rlm/print}          \\
 \texttt{rda}    & \texttt{application/octet-stream} & \texttt{base::save}     & \texttt{/ocpu/cran/MASS/data/cats/rda}          \\
 \texttt{rds}    & \texttt{application/octet-stream} & \texttt{base::saveRDS}  & \texttt{/ocpu/cran/MASS/data/cats/rds}          \\
 \texttt{json}   & \texttt{application/json}         & \texttt{jsonlite::toJSON}   & \texttt{/ocpu/cran/MASS/data/cats/json}      \\
 \texttt{pb}     & \texttt{application/x-protobuf}   & \texttt{RProtoBuf::serialize\_pb} & \texttt{/ocpu/cran/MASS/data/cats/pb} \\
 \texttt{tab}    & \texttt{text/plain}               & \texttt{utils::write.table}   & \texttt{/ocpu/cran/MASS/data/cats/tab}    \\
 \texttt{csv}    & \texttt{text/csv}                 & \texttt{utils::write.csv}    & \texttt{/ocpu/cran/MASS/data/cats/csv}     \\
 \texttt{png}    & \texttt{image/png}                & \texttt{grDevices::png}      & \texttt{/ocpu/tmp/\{key\}/graphics/1/png}    \\
 \texttt{pdf}    & \texttt{application/pdf}          & \texttt{grDevices::pdf}      & \texttt{/ocpu/tmp/\{key\}/graphics/1/pdf}     \\
 \texttt{svg}    & \texttt{image/svg+xml}            & \texttt{grDevices::svg}      & \texttt{/ocpu/tmp/\{key\}/graphics/1/svg}     \\ \bottomrule
\end{tabular}
\caption{Supported object export formats and corresponding \texttt{Content-type}}
\label{table:formats}
\end{table}

\subsubsection{Containers}

The root of the API is dynamic. It defaults to \texttt{/ocpu/} however system administrators can change this. Clients should make the \OpenCPU server address and root path configurable. In the examples we assume the default \texttt{/ocpu/} is used.



\begin{table}[H]
\centering
\def\arraystretch{1.5}%
\begin{tabular}{@{}lll@{}}
\toprule
\emph{Path} & \emph{Desription}                      & \emph{Examples}                \\ \midrule
\texttt{.}    & Package information                      & \texttt{/ocpu/cran/MASS/}               \\
\texttt{./R}    & Exported namespace objects             & \texttt{/ocpu/cran/MASS/R/}             \\
     &                                                   & \texttt{/ocpu/cran/MASS/R/rlm/print}    \\
\texttt{./data} & Data objects in the package (\HTTP \GET only)         & \texttt{/ocpu/cran/MASS/data/}          \\
     &                                                   & \texttt{/ocpu/cran/MASS/data/cats/json} \\
\texttt{./man}  & Manual pages in the package (\HTTP \GET only)         & \texttt{/ocpu/cran/MASS/man/}           \\
     &                                                   & \texttt{/ocpu/cran/MASS/man/rlm/html}   \\
\texttt{./*}    & Files in installation directory of the package      & \texttt{/ocpu/cran/MASS/NEWS}    \\
     &                                                   & \texttt{/ocpu/cran/MASS/scripts/}       \\ \bottomrule
\end{tabular}
\caption{The package container includes objects, data, manual pages and files.}
\label{table:packageapi}
\end{table}



\begin{table}[H]
\centering
\def\arraystretch{1.5}%
\begin{tabular}{@{}lll@{}}
\toprule
\emph{Path}          & \emph{Desription}                      & \emph{Examples}                \\ 
\midrule
\texttt{.}          & Session content list                                       & \texttt{/ocpu/tmp/\{key\}/}               \\
\texttt{./R}        & Objects created by the \RPC request                        & \texttt{/ocpu/tmp/\{key\}/R/}             \\
                    &                                                            & \texttt{/ocpu/tmp/\{key\}/R/mydata/json}  \\
\texttt{./graphics} & Graphics created by the \RPC request                       & \texttt{/ocpu/tmp/\{key\}/graphics/}      \\
                    &                                                            & \texttt{/ocpu/tmp/\{key\}/graphics/1/png} \\
\texttt{./source}   & Source code of \RPC request                                & \texttt{/ocpu/tmp/\{key\}/source/}       \\
\texttt{./stdout}   & \texttt{STDOUT} from by the \RPC request                   & \texttt{/ocpu/tmp/\{key\}/stdout/}       \\
\texttt{./console}  & Mixed source and \texttt{STDOUT} emulating console output  & \texttt{/ocpu/tmp/\{key\}/console/}     \\
\texttt{./files/*}  & Files saved to working dir by the \RPC request             & \texttt{/ocpu/tmp/\{key\}/files/myfile.xyz}       \\

                                                
\bottomrule
\end{tabular}
\caption{The session container includes objects, graphics, source, stdout and files.}
\label{table:sessionapi}
\end{table}







\subsection{RPC}

A \POST request in \OpenCPU invokes a remote procedure call (\RPC). Requests targeting a \emph{function} object result in a function call where the \HTTP parameters from the post body are mapped to function \emph{arguments}. Similarly, a \texttt{POST} targeting a \emph{script} results in execution of the script where \HTTP parameters are passed to the script interpreter. The current \OpenCPU implementation recognizes scripts by their file extension, and supports \R, \texttt{latex}, \texttt{markdown}, \texttt{Sweave} and \texttt{knitr} scripts. We use the term \RPC to refer to both remote function calls and remote script executions. One conceptual difference with a terminal interface is that in the \OpenCPU \API, the server determines the name and namespace that output of a function call is assigned to. The server includes a temporary \emph{key} in the \RPC response that serves the same role as a variable name and is used to reference the newly created resources in future requests. This key is private and the client should treat it confidentially.

\begin{table}[H]
\centering
\def\arraystretch{1.5}%
\begin{tabular}{@{}lll@{}}
\toprule
\emph{File extension} & \emph{Type}           & \emph{Interpreter}                   \\ \midrule
\texttt{file.r}         & \R       & \texttt{evaluate::evaluate}            \\
\texttt{file.tex}       & \Latex          & \texttt{tools::texi2pdf}               \\
\texttt{file.rnw}       & \texttt{knitr}/\texttt{sweave}   & \texttt{knitr::knit} + \texttt{tools::texi2pdf} \\
\texttt{file.md}        & \texttt{markdown}       & \texttt{knitr::pandoc}                 \\
\texttt{file.rmd}       & \texttt{knitr markdown} & \texttt{knitr::knit} + \texttt{knitr::pandoc}   \\
\texttt{file.brew}      & \texttt{brew}           & \texttt{brew::brew}                    \\ \bottomrule
\end{tabular}
\caption{A \POST request targeting a file with a recognized file extension results in an execution of the script}
\label{table:scripts}
\end{table}

Besides the return value, the server stores graphics, files, warnings, messages and \texttt{stdout} that were created during the function call. These contents can be listed and retrieved using the same key as for the object. In \R, the function call itself is also an object which is added to the collection for reproducibility purposes. Objects on the system are non mutable and therefore the client can not change or overwrite existing keys or objects. For function calls that modify the state of an object, the server creates a copy of the modified object under a new name and leaves the original unaffected.

\subsubsection{RPC Arguments}

Arguments to a remote function call can be posted using one of several methods. A data interchange format such as \JSON or \texttt{Protocol Buffers} can be used to directly post data structures such as lists, vectors, matrices or data frames. Alternatively the client can reference an existing object on the server using a key or name. The server automatically resolves keys and converts data structures into native objects which are used as arguments in the function call. Hence the function that is called is not restricted to a particular data interchange format. Files can also be used as arguments within remote function calls. Any files contained in a \texttt{multipart/form-data}  post request will be copied to the working directory before the function call is executed. The server then sets the argument in the function call to the \texttt{filename}. Thereby we can remotely call functions with a file argument using standard \HTML form submission.

\begin{table}[H]
\def\arraystretch{1.5}%
\begin{tabular}{@{}llllll@{}}
\toprule
\emph{Content-type}                      & \emph{Primitives} & \emph{Data structures}  &  \emph{Raw code} & \emph{File} & \emph{Temp key} \\ \midrule
\texttt{multipart/form-data}               & OK         & OK (inline \texttt{json}) & OK       & OK   & OK            \\
\texttt{application/x-www-form-urlencoded} & OK         & OK (inline \texttt{json}) & OK       & -    & OK            \\
\texttt{application/json}                  & OK         & OK               & -        & -    & -             \\
\texttt{application/x-protobuf}            & OK         & OK               & -        & -    & -             \\ \bottomrule
\end{tabular}
\caption{Accepted request \texttt{Content-types} and supported argument formats}
\label{table:arguments}
\end{table}

The current implementation supports several \texttt{Content-type} formats for passing arguments to a remote function call within a \POST request. These include \texttt{application/x-www-form-urlencoded}, \texttt{multipart/form-data}, \texttt{application/json} and \texttt{application/x-protobuf}. Each parameter or top level object within a \POST requests contains an argument value. These values can be specified one of several argument formats as listed in table \ref{table:arguments}. Not every \texttt{Content-type} supports each argument format. 

\subsubsection{Reproducibility}

The \OpenCPU specification makes reproducibility an integrated part of the \API interaction. For each \RPC request the server needs to store the call, script, arguments or input files, in addition to results. The same key that is used to retrieve output resources such as objects or graphics, can be used to retrieve the input resources or automatically replicate the entire computation. Hence for each output resource on the system, clients can lookup the code, data, warnings and packages that were involved in the creation of an object. Thereby results can easily and automatically be recalculated, which forms a powerful foundation for reproducible practices. However this system can also be used for other purposes. For example, if a procedure fetches dynamic data from an external resource to generate a model or plot, we can use reproduction to \emph{update} the model or plot with the latest data.

