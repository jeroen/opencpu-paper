% !TEX root = main.tex

%\section{Conclusion}

%The parts that were left out of the interface specification are just as interesting as the \API itself. For example the \RPC mechanism implies that concerns of managing processes, hardware and security are at the discretion of the server. Moreover, because clients can specify the data interchange format for retrieving an arbitrary resource, the server needs to implement mappings between the various resource types and data interchange formats. And the fact that computational exceptions are returned using \HTTP status codes implies that error handling is the responsibility of the client. These are some examples of how the \API implicitly gives rise to the boundaries for separation concerns and suggest a minimal structure for embedded scientific computing. 

%The end goal of this paper is to work towards an interface definition for embedded scientific computing. An interface is the embodiment of separation of concerns and serves as a contract that formalizes the boundary across which separate components exchange information. The interface definition describes the concepts and operations that components agree upon to cooperate and how the communication is arranged. Through the interface we specify the functionality that a server has to implement, which parts of the interaction are fixed and which choices are specifically left at the discretion of the implementation. Ideally the specification should provide sufficient structure to develop clients and server components for scientific computing while minimizing limitations on how these can be implemented. An interface that carefully isolates components along the lines of domain logic allows developers to focus on their expertise using their tools of choice. It gives clients a universal point of interaction to integrate statistical methods without understanding the actual computing, and allows statisticians to implement methods for use in applications without knowing specifics about the application layer.